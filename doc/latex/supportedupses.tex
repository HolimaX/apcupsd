\label{Supported-UPSes}
\section*{Supported UPSes and Cables}
\index{Supported UPSes and Cables}
\index{UPSes!Supported}
\index{Cables!Supported}
\addcontentsline{toc}{section}{Supported UPSes and Cables}

You can generally count on your UPS being supported if it has either an
Ethernet-connected SNMP interface or a USB interface with an APC-supplied
cable.  

\label{upstypes}
The ``UPSTYPE Keyword'' field is the value you will put in your
/etc/apcupsd/apcupd.conf file to tell apcupsd what type of UPS you have. 
We'll describe the possible values here, because they're a good way to explain
your UPS's single most important interface property {--} the kind of protocol
it uses to talk with its computer.  

\label{index-UPSTYPE-10}
\label{index-Keywords_002c-USBTYPE-11}

\begin{description}

\item [apcsmart]
   \index{apcsmart }
   An APCSmart UPS and its computer also communicate through an RS232C serial
connection, but they actually use it as a character channel (2400bps, 8 data
bits, 1 stop bit, no parity) and pass commands back and forth in a primitive
language (see 
\ilink{APC smart protocol}{APC-smart-protocol}) resembling
modem-control codes.  The different APC UPSes all use closely related
firmware, so the language doesn't vary much (later versions add more
commands).  This class of UPS is in decline, rapidly being replaced in APC's
product line by USB UPSes.  

\item [usb]
   \index{usb }
   A USB UPS speaks a universal well defined control language over a USB wire. 
Most of APC's lineup now uses this method as of late 2003, and it seems likely
to completely take over in their low- and middle range.  Other manufacturers
(Belkin, Tripp-Lite, etc.) are moving the same way, though with a different
control protocol for each manufacturer.  As long as USB hardware can be
mass-produced more cheaply than an Ethernet card, most UPSes are likely to go
this design route. Please note that even if you have a USB UPS, if you use a
serial cable with it (as can be supplied by APC), you will need to configure
your UPS as {\bf apcsmart} rather than {\bf usb}.  

\item [net]
   \index{net }
   This is the keyword to specify if you are using your UPS in Slave mode (i.e.
the machine is not directly connected to the UPS, but to another machine which
is), and it is connected to the Master via an ethernet connection. You must
have apcupsd's Network Information Services NIS turned on for this mode to
work. It is a much simpler form of running a Slave than the old Master/Slave
code.  

\item [snmp]
   \index{snmp }
   SNMP UPSes communicate via an Ethernet NIC and firmware that speaks Simple
Network Management Protocol.  

\item [dumb]
   \index{dumb }
   A dumb or voltage-signaling UPS and its computer communicate through the
signal lines on an RS232C serial connection.  Not much can actually be
conveyed this way other than an order to shut down. Voltage-signaling UPSes
are obsolete; you are unlikely to encounter one other than as legacy hardware.
If you have a choice, we recommend you avoid simple signalling UPSes.  
\end{description}

The table shown below lists the APC model supported, and the possible kewords
that you would use in the configuration with the listed cables.
See below for more details on the keywords.
Some of the
models, particularly USB enabled models, can be run in multiple modes, so they
may appear more than once in the table. APC is putting out new models at a
furious rate, and so it is very likely that your model is not listed in the
table. If it is USB enabled, it will probably work in USB mode. Please note
that some of these new models are extremely inexpensive, so they are stripped
down versions of more expensive units, and as such they do not offer as many
features, so some of the example output you see elsewhere in this manual may
not be available with your unit. 

\label{type_005ftable}
\label{index-UPSTYPE_002c-table-12}

\addcontentsline{lot}{table}{Supported UPS Models}
\begin{longtable}{|p{3in}|p{1in}|p{1in}|p{2in}}
\hline
\multicolumn{1}{|c| }{{\bf APC Model}} & 
\multicolumn{1}{c| }{{\it UPSTYPE Keyword}} & 
\multicolumn{1}{c| }{{\it UPSCABLE keywords for Cables Supported}} & 
\multicolumn{1}{c| }{{\it Status}  } \\
\hline
{BackUPS CS/ES (serial mode)} & {apcsmart} & {smart (note: using Smart Custom
RJ45) the new Back-UPS RS 500 models are reported NOT to work with this
cable.} & {Supported 
 } \\
\hline
{BackUPS Pro, Smarter BackUPS Pro} & {apcsmart} & {940-0095A} & {Supported 
 } \\
\hline
{SmartUPS, SmartUPS VS (It has not been confirmed that the cable shipped with
the VS is a 940-0095.), PowerStack 450, Matrix UPS, ShareUPS Advanced Port} &
{apcsmart} & {smart (note: using Smart-Custom), 940-0024C } & {Supported 
 } \\
\hline
{BackUPS CS USB, Pro USB, ES USB, RS/XS 1000, RS/XS 1500, and probably other
USB models} & {usb} & {usb (note: using APC cables 940-0127A/B/C)} &
{Supported in version \gt{}=3.9.8 
 } \\
\hline
{SmartUPS USB, BackUPS Office USB, and any other USB UPS} & {usb} & {usb
(note: using APC cable, no number)} & {Supported, version \gt{}=3.9.8 
 } \\
\hline
{All SNMP-capable models} & {snmp} & {ether} & {Supported 
 } \\
\hline
{BackUPS} & {dumb} & {simple (note: using Simple-Custom (This cable is not an
APC product.  You have to build it yourself using the instructions in 
\ilink{Cables}{Cables}.), 940-0020B, 940-0020C, 940-0119A,
940-0023A} & {Supported 
 } \\
\hline
{BackUPS Office, BackUPS ES} & {dumb} & {940-0119A} & {Supported 
 } \\
\hline
{BackUPS CS and possibly ES models (serial mode)} & {dumb} & {940-0128A} &
{Supported 
 } \\
\hline
{ShareUPS Basic Port} & {dumb} & {940-0020B, 940-0020C, 940-0023A} &
{Supported }
\hline

\end{longtable}

\label{upstypes}
The ``UPSTYPE Keyword'' field is the value you will put in your
/etc/apcupsd/apcupd.conf file to tell apcupsd what type of UPS you have. 
We'll describe the possible values here, because they're a good way to explain
your UPS's single most important interface property {--} the kind of protocol
it uses to talk with its computer.  

\label{index-UPSTYPE-10}
\label{index-Keywords_002c-USBTYPE-11}

\begin{description}

\item [apcsmart]
   \index{apcsmart }
   An APCSmart UPS and its computer also communicate through an RS232C serial
   connection, but they actually use it as a character channel (2400bps, 8 data
   bits, 1 stop bit, no parity) and pass commands back and forth in a primitive
   language (see 
   \ilink{APC smart protocol}{APC-smart-protocol}) resembling   
   modem-control codes.  The different APC UPSes all use closely related
   firmware, so the language doesn't vary much (later versions add more
   commands).  This class of UPS is in decline, rapidly being replaced in APC's
   product line by USB UPSes.  

\item [usb]
   \index{usb }
   A USB UPS speaks a universal well defined control language over a USB wire. 
   Most of APC's lineup now uses this method as of late 2003, and it seems likely
   to completely take over in their low- and middle range.  Other manufacturers
   (Belkin, Tripp-Lite, etc.) are moving the same way, though with a different
   control protocol for each manufacturer.  As long as USB hardware can be
   mass-produced more cheaply than an Ethernet card, most UPSes are likely to go
   this design route. Please note that even if you have a USB UPS, if you use a
   serial cable with it (as can be supplied by APC), you will need to configure
   your UPS as {\bf apcsmart} rather than {\bf usb}.  

\item [net]
   \index{net }
   This is the keyword to specify if you are using your UPS in Slave mode (i.e.
 the machine is not directly connected to the UPS, but to another machine which
 is), and it is connected to the Master via an ethernet connection. You must
 have apcupsd's Network Information Services NIS turned on for this mode to
 work. It is a much simpler form of running a Slave than the old Master/Slave
 code.  

\item [snmp]
   \index{snmp }
   SNMP UPSes communicate via an Ethernet NIC and firmware that speaks Simple
Network Management Protocol.  

\item [dumb]
   \index{dumb }
   A dumb or voltage-signaling UPS and its computer communicate through the
signal lines on an RS232C serial connection.  Not much can actually be
conveyed this way other than an order to shut down. Voltage-signaling UPSes
are obsolete; you are unlikely to encounter one other than as legacy hardware.
If you have a choice, we recommend you avoid simple signalling UPSes.  
\end{description}
